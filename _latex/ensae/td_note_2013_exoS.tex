L'objectif de cet exercice est de colorier l'espace situ� entre les deux spirales de la figure~\ref{double_spirale_2013}.


\begin{figure}[ht]
\begin{center}\begin{tabular}{|c|}\hline 
\includegraphics[width=8cm]{../python_examen/image/double_ellipse.png} \\ \hline
\end{tabular}
\end{center}
\caption{Double spirale}
\label{double_spirale_2013}
\end{figure}

Le code qui a servi � construire ces deux spirales vous est fourni en pi�ce jointe\footnote{lien \httpstyle{http://www.xavierdupre.fr/enseignement/complements\_site\_web/td\_note\_2013\_novembre\_2012\_exoM.py}}. Ce code vous est donn�, aucune question ne vous sera pos�e dessus. Il est pr�sent � la fin de l'�nonc� de l'exercice. \textbf{Afin d'�viter son inclusion (et son impression), votre programme devra imp�rativement commencer par~:}

\begin{verbatimx}
#coding:latin-1
import exoS
matrice = exoS.construit_matrice(100)
\end{verbatimx}

Pour visualiser la matrice et obtenir la figure~\ref{double_spirale_2013}, il faut �crire~:
\begin{verbatimx}
exoS.dessin_matrice(matrice)   # cette fonction place un point bleu pour une case contenant 1,
                               # rouge pour une case contenant 2,
                               # rien si elle contient 0
\end{verbatimx}

Au d�part la matrice retourn�e par la fonction \codes{construit\_matrice} contient soit~0 si la case est vide, soit~1 si cette case fait partie du trac� d'une spirale. L'objectif de cet exercice est de colorier une zone de la matrice.

\exequest Ecrire une fonction qui prend comme entr�e une matrice, deux entiers et qui retourne la liste des voisins parmi les quatre possibles pour lesquels la matrice contient une valeur nulle. On fera attention aux bords du quadrillage. (2~points)

\begin{verbatimx}
def voisins_a_valeur_nulle ( matrice, i, j ) :
    resultat = [ ]
    # ...
    return resultat
\end{verbatimx}


\exequest En utilisant la fonction pr�c�dente, �crire une fonction qui re�oit une liste de points de la matrice contenant une valeur nulle et qui retourne tous les voisins possibles contenant une valeur nulle pour tous les points de la liste. (2~points)

\begin{verbatimx}
def tous_voisins_a_valeur_nulle ( matrice, liste_points ) :
    resultat = [ ]
    # ...
    return resultat
\end{verbatimx}

\exequest On a tous les �l�ments pour �crire l'algorithme de coloriage~:
\begin{enumerate}
\item On part d'un point $(i0,j0)$ qui fait partie de la zone � colorier, qu'on ins�re dans une liste \codes{acolorier = [(i0,j0)]}.
\item Pour tous les points $(i,j)$ de la liste \codes{acolorier}, on change la valeur de la case $(i,j)$ de 0 � 2.
\item Pour tous les points $(i,j)$ de la liste \codes{acolorier}, on regarde les quatre voisins $(i,j+1), (i,j-1), (i+1,j), (i-1,j)$. Si la matrice contient 0 pour un de ces voisins, on l'ajoute � la liste et on retourne l'�tape pr�c�dente tant que cette liste n'est pas vide.
\end{enumerate}

En utilisant la fonction pr�c�dente, �crire une fonction qui colorie la matrice � partir d'un point $(i0,j0)$. (3~points)

\begin{verbatimx}
def fonction_coloriage( matrice, i0, j0) :
    # ...
\end{verbatimx}


\exequest Tester la fonction pr�c�dente avec le point de coordonn�es $(53,53)$ puis afficher le r�sultat avec la fonction \codes{dessin\_matrice}. (1~point)
